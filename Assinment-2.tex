\documentclass{article}
\title{THE 18.821 MAHTEMATICS PROJECT LAB REPORT \\[REPLACE THIS WITH YOUR OWN SHORT DESCRIPTIVE TITLE!]\\
	\vspace{1cm}X. BURPS. P.GURPS}
\date{\today}

\usepackage{amsmath}
\usepackage{xcolor}
\usepackage{lipsum}

 
\begin{document}
\maketitle

 	 \begin{abstract}
 	  ABSTRACT. This is a L A TEX template for 18.821, which you can
 	  use for your own reports.
 	 \end{abstract}
  \centering \section{INTRODUCTION}
  This brief document shows some examples of the use of LATEX and
  indicates some special features of the Math Lab report style. The
  course website contains links to several LATEX manuals.
  End the introduction by describing the contents of the paper sec­
  tion by section, and which team member(s) wrote each of them. For
  instance, Section 6 discusses referencing, and is written by P. Gurps.
	
	\section{LATEX Examples}

	Here are some ways of producing mathematical symbols. Some are
	package which
	pre-defined either in LATEX or in the AMS a
	loads. For instance, sums and integrals,$\sum_{i=1}^{n} i=1 1 = n, \int^n_0 x dx= n^2/2.$
	We’ve defined a few other symbols at the start of the document, for
	instance N, Q, Z, R. You can make marginal notes for yourself or your
	co-authors like this:\\
	Unfinished here?
	If you want to typeset equations, there are many choices, with or
	without numbering:\\
	 $\int^1_0 xdx=1/2$  \vspace{1mm}
	 \\
	or \hspace{5mm}    $\sum_{i=1}^{\infty}=-\frac{1}{12}$
	\\
	or \hspace{1cm}
		$1-1+1-\ldots=\frac{1}{2}$.
		\\ \vspace{1cm}
	\caption{X. BURPS, P. GURPS}
		
%		image here...!
	
	If you want a number for an equation, do it like this:\\
	$(1) \lim_{n\to\infty} \sum_{k=1}^{n}\frac{1}{k^2}=\frac{\pi}{6}.$
	\\
	This can then be referred to as (1), which is much easier than keeping
	track of numbers by hand. To group several equations, aligning on the
	= sign, do it like this:
	\\
	 $x_1+2x_2+3x_3=7$ \\
	 $y=mx+c$ \\
	 $=4x-9.$ \\
	 You can easily embed hyperlinks into the output .pdf document:
	 {\textcolor{blue}{click here}} for example. \\
	 
	 \section{IMAGES}
	 Figure 1 is an example of a .pdf image put into a floating environ­
	 ment, which means LaTeX will draw it wherever there’s enough space
	 left in your manuscript. Look at the .tex original to see how to insert
	 a figure like this. \\
	 
	 \section{Theorems and such}
	 An example of a “conjecture environment” is given below, in Con­
	 jecture 4.1. Theorems, lemmas, propositions, definitions, and such all
	 use the same command with the appropriate name changed. In fact,
	 \\
	 \caption{THE 18.821 REPORT}
	 
	if you look at the top of this .tex file, you can see where we’ve defined
	these environments.
	\\ \vspace{3mm}
	\textbf{Conjecture 4.1}
	(Vaught’s Conjecture). Let T be a countable com­
	plete theory. If T has fewer than 2 ℵ 0 many countable models (up to
	isomorphism), then it has countably many countable models.\\ \vspace{3mm}
	\textbf{Theorem 4.2.} When it rains it pours.\\ Proof. Well, yes.
	\\
	\section{Filetypes used by LaTeX}

You will write your text as a .tex file using any text editor (though
WYSIWYG editors are troublesome). Traditionally one then runs
L A TEX and obtains a .dvi file, which can be viewed on the screen using a
dvi viewer. To include images, and then prepare the file for printing or
submission, one typically translates the .dvi into either .ps (Postscript)
or .pdf (Adobe PDF).
Your report will be submitted as a .pdf document. The pdflatex
command produces a .pdf file directly from a .tex file. This command
works well with included .pdf files, but does not handle .eps files.
An .eps file can be converted to a .pdf file by viewing it and saving
as a .pdf file, or by ps2pdf filename.eps, which produces
filename.pdf. Under MikTeX with WinEdt, all necessary commands
will appear under “Accessories” in the WinEdt menu.
Finally, Matlab can be made to produce .eps files by typing
print -deps filename
at the prompt. 
\\ \vspace{3mm}

\section{Quoting sources}
In your work, keep notes of the literature you’ve used, including
websites. Cite the references you use; failure to do so constitutes pla­
giarism. Every bibliography item should be referenced somewhere in
the paper. Quote as precisely as possible: [1, pages 76–78] rather than
[1]. [2] was a useful background reference, too.
	\\ \vspace{3mm}
      \textbf{References}
      \\ \vspace{3mm}
      [1] Gurps, P., Care and feeding of maths professors. Cambridge Univ. Press, 2008.
      [2] Burps, X. Terrors and errors of project lab. Journal of Wildlife and Conserva­
      tion 21 (2008), 112–134.

	\\ \vspace{3mm}
	\textbf{Appendix}
	\\ \vspace{3mm}
	Appendices are useful for putting in code or data.
	\\ \vspace{1cm}
	\textbf{MIT OpenCourseWare}\\
	\textcolor{blue}{http://ocw.mit.edu} 
	\\
	18.821 Project Laboratory in Mathematics
	Spring 2013
	\\ \vspace{2cm}
	For information about citing these materials or our Terms of Use, visit:\textcolor{blue}{http://ocw.mit.edu/terms.}
\end{document}